\section{M�todo y Fases de trabajo}

Para el desarrollo de la herramienta se ha optado por utilizar la metodolog�a gen�rica descrita por el \textbf{Proceso Unificado de Desarrollo} (PUD). 
El PUD \cite{jac00} es una evoluci�n del Proceso Unificado de Rational (RUP), que define un ``conjunto de
actividades necesarias para transformar los requisitos de usuario en un sistema software''. Es
un marco gen�rico que puede especializarse para una gran variedad de sistemas de software,
para diferentes �reas de aplicaci�n, diferentes tipos de organizaciones, diferentes niveles de
aptitud y diferentes tama�os de proyectos. 

Las principales caracter�sticas del PUD son:

\begin{milista}

	\item \textbf{Dirigido por casos de uso}. Un caso de uso representa un requisito funcional al cu�l el sistema debe dar soporte para proporcionar un resultado de valor al usuario. Los casos de uso gu�an el proceso de desarrollo (dise�o,
implementaci�n, y prueba), ya que bas�ndose en los casos de uso, los desarrolladores
crean una serie de modelos para poder llevarlos a cabo. Todos los casos de uso juntos constituyen el \textbf{modelo de casos de uso}.
	\item \textbf{Centrado en la arquitectura}. Un sistema software puede contemplarse desde varios
puntos de vista. Por tanto, la arquitectura software incluye los aspectos est�ticos y
din�micos m�s significativos del sistema y debe estar profundamente relacionada con
los casos de uso ya que debe permitir el desarrollo de los mismos.
	\item \textbf{Iterativo e incremental}. El esfuerzo de desarrollar un proyecto de software se divide en partes m�s
peque�as, llamadas \textbf{mini-proyectos}. Cada mini-proyecto es una \textbf{iteraci�n} que resulta en un \textbf{incremento}.
Las iteraciones deben estar controladas y deben seleccionarse y
ejecutarse de una forma planificada, siguiendo el esquema \textit{requisitos, an�lisis, dise�o, implementaci�n y pruebas}, que es conocido como 
\textbf{flujo de trabajo}.
En cada iteraci�n, los desarrolladores identifican y especifican los casos de uso relevantes, crean un dise�oo utilizando la arquitectura seleccionada como gu�a, implementan el dise�o mediante componentes y verifican que los componentes satisfacen los casos de uso.

\end{milista}

Debido al enfoque iterativo e incremental que caracteriza al PUD, �ste se divide en ciclos, donde cada ciclo se compone de cuatro fases y �stas, a su vez, se dividen en iteraciones que siguen el flujo de trabajo \textit{requisitos, an�lisis, dise�o, implementaci�n y pruebas} (ver Figura \ref{fig:pud}).\\

\imagen{pud}{0.65}{Fases y flujo de trabajo del PUD}{fig:pud}

\indent Las cuatro fases que constituyen un ciclo son:

\begin{milista}

	\item \textbf{Iniciaci�n}. Se obtiene un modelo de casos de uso simplificado, se identifican
riesgos potenciales y se estima el proyecto de manera aproximada.
	\item \textbf{Elaboraci�n}. Se especifican en detalle la mayor�a de los casos de
uso del producto software y se dise�a la arquitectura, obteniendo la l�nea base de la arquitectura.
	\item \textbf{Construcci�n}. Se crea el producto en base a las dos fases anteriores. La l�nea base de la arquitectura
crece hasta convertirse en el sistema completo.
	\item \textbf{Transici�n}. Implica la correcci�n de errores y el mantenimiento del
sistema.

\end{milista}

