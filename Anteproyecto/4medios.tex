\section{Medios que se pretenden utilizar}

Para el desarrollo del PFC se utilizar�n una serie de medios software, tanto para la especificaci�n y definici�n del sistema, como para su implementaci�n. Dichos medios son: 

\begin{milista}
	\item \textbf{Eclipse} \cite{eclipse} (\textit{versi�n 3.6 Helios}), ya que la herramienta a desarrollar es un plug-in para dicho IDE.
	\item \textbf{Java} \cite{java}, en su �ltima versi�n (\textit{versi�n 6 Update 23}).
	\item \textbf{Apache Commons Configuration} \cite{apacheCommons}, para leer archivos de configuraci�n de diferentes tipos (\textit{properties}, \textit{xml}, etc�tera).
	\item \textbf{JAXB (Java Architecture for XML Binding)} \cite{jaxb}, para poder mapear y exportar clases Java a un fichero XML.
	\item \textbf{Zest: The Eclipse Visualization Toolkit} \cite{zest}, para visualizar el conocimiento en forma de grafo o �rbol.
	\item \textbf{JFreeChart} \cite{jfreechart}, para mostrar gr�ficos.
	\item \textbf{MySQL} \cite{mysql}, para la elaboraci�n de la base de datos de conocimiento con la que debe comunicarse la herramienta. 
	\item \textbf{Hibernate} \cite{hibernate}, para gestionar la persistencia.
	\item \textbf{Visual Paradigm} \cite{visualparadigm}, para la realizaci�n de los diferentes diagramas de la herramienta. 
	\item \textbf{LaTeX} \cite{latex}, para la elaboraci�n de la documentaci�n. 
\end{milista}