\chapter{Motivaci�n y Objetivos del Proyecto}
\label{ch:motivacionyobjetivos}

% Anteproyecto m�s alg�n detalle

% Concretar y exponer el problema a resolver describiendo el entorno de trabajo, la situaci�n y detalladamente qu� se pretende obtener. Limitaciones y condicionantes a considerar para la resoluci�n del problema (lenguaje de construcci�n, equipo f�sico, equipo l�gico de base o de apoyo, etc.).

\section{Motivaci�n}

\section{Objetivos}

%\section{Entorno y requisitos}

% Para poder obtener una interfaz gr�fica de usuario de estas caracter�sticas ser� necesario crear dos anotaciones personalizadas de Java

% anotaciones
% parser de anotaciones
% generaci�n de ficheros xml con jaxb para los perfiles , que ser� persistente en formato XML (\textit{Extensible Markup Language}), 
% introspecci�n o reflection
% motor gr�fico con JOGL
% JSON para la entrada del motor gr�fico
% applet y comunicaci�n bidireccional mediante javascript
% seguridad y acceso a p�ginas moderado con spring security
% instanciaci�n de acciones y dao mediante spring
% acciones mediante struts
% men�s mediante struts-menu
% jquery
% json para ajax
% google maps