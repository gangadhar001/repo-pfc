% Se debe recordar que aunque en el tipo de documento report se puede incluir un abstract (resumen). En la clase book este entorno no est� definido.
%====================================
%====================================
%.... RESUMEN (opcional, 1 p�g.)
\selectlanguage{spanish}%
\begin{abstract}

En las �ltimas d�cadas, el fen�meno de la globalizaci�n est� afectando al modelo de negocio de las empresas, evolucionando hacia un mercado global que busca reducir los costes, aumentar la productividad y la ventaja competitiva. \\
\indent Las empresas dedicadas al desarrollo de software no son ajenas a este fen�meno, y tambi�n se est�n adaptando para desarrollar el software de manera distribuida, en diferentes equipos de desarrollo dispersos por cualquier pa�s del mundo. Es lo que se conoce como \textit{Desarrollo Global de Software (DGS)}. \\
\indent Este paradigma de desarrollo software introduce un gran n�mero de ventajas para las empresas que lo siguen, pero tambi�n introduce una serie de dificultades y desaf�os, asociados a las distancias geogr�ficas, temporales y socio-culturales. Una de las principales dificultades aparece con la Gesti�n del Conocimiento y de Decisiones, ya que se produce informaci�n desde muchas y diversas fuentes, lo que dificulta su gesti�n, almacenamiento y reutilizaci�n. \\
\indent Para paliar algunos de estos desaf�os, se propone desarrollar un sistema que permita dar soporte a la gesti�n de decisiones en proyectos software en el contexto de desarrollo global. Por tanto, dicho sistema debe permitir la creaci�n, almacenamiento, recuperaci�n y transmisi�n de decisiones abordadas en un proyecto software, realizado de manera deslocalizada. Adem�s, debe permitir gestionar
tambi�n los proyectos software y reutilizar dichas decisiones tomadas en proyectos previos en nuevos proyectos con caracter�sticas similares. 

\end{abstract}

%====================================
%====================================
%.... ABSTRACT (opcional, 1 p�g.)
% Aqu� se muestra un ejemplo completo en el que se a�ade una versi�n en ingl�s del resumen.
\selectlanguage{english}% Selecci�n de idioma ingl�s.
\begin{abstract}
... english version for the abstract ...
\end{abstract}

\selectlanguage{spanish}% Para el resto del documento del idioma es espa�ol.
%====================================
%====================================
%.... AGRADECIMIENTOS (opcional, 1 p�g. o m�s, recomendable hacerloe en una s�lo)
% Los agradecimientos son m�s extensos que la dedicatoria como se muestra en el ejemplo.
\chapter*{Agradecimientos} % Opci�n con * para que no aparezca en TOC
Escribir agredecimientos