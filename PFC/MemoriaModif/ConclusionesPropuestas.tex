\chapter{Conclusiones y Propuestas}
\label{ch:conclusionesypropuestas}

En las secciones posteriores de este cap�tulo se desarrollan las conclusiones obtenidas tras el desarrollo del PFC, as� como algunas propuestas para realizar en un futuro.

\section{Conclusiones}

Tras haber realizado y finalizado el desarrollo del PFC, se ha podido comprobar como gracias a utilizar una metodolog�a de desarrollo iterativa e incremental, como es el PUD, el sistema ha ido aumentando en funcionalidad de manera progresiva, facilitando las tareas de an�lisis, dise�o, implementaci�n y pruebas de cada iteraci�n, que daban como resultado un nuevo incremento en la funcionalidad de la aplicaci�n.

Adem�s, gracias a este car�cter iterativo del PUD y a su divisi�n en fases, se ha podido resolver de una manera sencilla y r�pida la incorporaci�n de los nuevos requisitos que fueron detectados al comienzo de la fase de Elaboraci�n, puesto que fueron identificados en iteraciones muy tempranas del ciclo de vida y, por tanto, no causaron un retraso ni impacto importante en el desarrollo del sistema.

En cuanto a las tecnolog�as y \textit{frameworks} utilizados, Hibernate ha permitido gestionar las operaciones relacionadas con las bases de datos de una manera transparente, permitiendo cambiar el SGBD utilizado sin afectar de ning�n modo al sistema desarrollado. 

Por otra parte, la elecci�n de RMI ha supuesto una serie de ventajas a la hora de realizar el sistema distribuido: 

\begin{itemize}
	\item Los objetos remotos pueden ser manejados como si fueran locales.
	\item Gracias a la utilizaci�n de interfaces para comunicar los subsistemas, �stos son totalmente independientes de su implementaci�n, y se pueden extender con nuevo m�todos de una manera sencilla.
	\item El servicio de registro de RMI, \textit{rmiregistry}, permite f�cilmente localizar e invocar los objetos remotos por su nombre.
	\item Al estar basado e integrado en Java, resulta sencillo y transparente al programador implementar el modelo de objetos distribuidos.
\end{itemize}

Como principal inconveniente de utilizar RMI, se puede destacar el problema de utilizarlo junto a Hibernate, ya al serializar y transferir los objetos remotos, estas referencias ya no son las mismas que las referencias que mantiene Hibernate para gestionar su modelo de objetos, por lo que se encontraron problemas a la hora de, por ejemplo, modificar objetos en la base de datos cuando el cliente hab�a realizado cambios en ellos. Sin embargo, este problema qued� solucionado clonando los objetos que fueron serializados por RMI y que deben modificarse en la base de datos, usando Hibernate.

Para concluir, con el desarrollo de este PFC se han adquirido conocimientos acerca de los temas que en �l se abordan, gracias a la revisi�n de la literatura existente, especialmente en la gesti�n de decisiones seg�n \textit{Rationale}, y a la comprensi�n de un m�todo de Inteligencia Artificial, como es el CBR, pudiendo dise�ar e implementar un algoritmo de CBR en el sistema. Adem�s, tambi�n se ha adquirido o se ha profundizado en el conocimiento y manejo de las herramientas y tecnolog�as utilizadas para la construcci�n de la aplicaci�n, destacando el uso de RMI, Hibernate, iText, JUNG y c�mo dise�ar y generar interfaces gr�ficas de usuario extensibles, flexibles y adaptables, utilizando para ello, adem�s de en otros casos, el API de reflexi�n de Java.

%% INDEPENDENCIA CLIENTE SRVIDOR

\section{Trabajo Futuro}


% Eventos y workflow, con hitos
% Parte Web
% BIRT