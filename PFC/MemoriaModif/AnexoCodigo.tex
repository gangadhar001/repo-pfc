\chapter{C�digo Fuente}
\label{appendix:codigo}

En las secciones que componen este anexo se muestran fragmentos de c�digo destacables en el sistema desarrollado.

\section{Implementaci�n de los patrones Fachada} \label{appendix:fachadas}

A continuaci�n se muestran los m�todos de cada una de las interfaces, o fachadas, de los subsistemas cliente y servidor.

\subsection{Fachada para el sistema servidor}

\texttt{\lstinputlisting[caption=Implementaci�n de la interfaz del sistema servidor, breaklines=true, label=list:facadeServer, inputencoding=latin1, style=JavaStyle]{Codigo//facadeServer.java}}

\subsection{Fachada para el sistema cliente}

%\texttt{\lstinputlisting[caption=Implementaci�n de la interfaz del sistema cliente, breaklines=true, label=list:facadeClient, inputencoding=latin1, style=JavaStyle]{Codigo//facadeClient.java}}


\section{Invocaci�n de un Servicio Web} \label{appendix:yahoo}

\texttt{\lstinputlisting[caption=Invocaci�n del servicio Web Yahoo! PlaceFinder, breaklines=true, label=list:GeoCoder, inputencoding=latin1, style=JavaStyle]{Codigo//GeoCoder.java}}

\section{Controlador de clientes} \label{appendix:remote}

\texttt{\lstinputlisting[caption=Fragmento de c�digo del controlador de clientes, breaklines=true, label=list:observador, inputencoding=latin1, style=JavaStyle]{Codigo//observadorClientes.java}}


\section{Soporte multi-hilo} \label{appendix:hilos}

\texttt{\lstinputlisting[caption=Soporte multi-hilo para actualizar el estado de clientes, breaklines=true, label=list:hilos, inputencoding=latin1, style=JavaStyle]{Codigo//hilos.java}}

\section{Algoritmo \textbf{NN} de CBR} \label{appendix:CBR}

\texttt{\lstinputlisting[caption=Fragmento de c�digo para el algoritmo \textit{NN} del CBR, breaklines=true, label=list:cbr, inputencoding=latin1, style=JavaStyle]{Codigo//cbr.java}}


\section{Generaci�n de PDF} \label{appendix:pdf}

\texttt{\lstinputlisting[caption=Fragmento de c�digo para la generaci�n de documentos PDF, breaklines=true, label=list:generarPDF, inputencoding=latin1, style=JavaStyle]{Codigo//generarPDF.java}}

\section{Generaci�n de un \textit{dataset} para un gr�fico} \label{appendix:dataset}

\texttt{\lstinputlisting[caption=Fragmento de c�digo para la generaci�n de \textit{datasets}, breaklines=true, label=list:dataset, inputencoding=latin1, style=JavaStyle]{Codigo//dataset.java}}

