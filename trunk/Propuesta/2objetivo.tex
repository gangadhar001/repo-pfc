\section{Objetivo} \label{objetivo}

En el desarrollo global de software, al contar con equipos de desarrollo distribuidos por todo el mundo, es de vital importancia la comunicaci�n y sincronizaci�n entre dichos equipos para que el sistema software pueda ser desarrollado exitosamente en el plazo de tiempo previsto. 

Uno de los aspectos a destacar en el desarrollo global del software es la gesti�n del conocimiento, ya que se debe poder registrar, almacenar y organizar todo el conocimiento (ideas, decisiones, experiencias, etc.) de los diferentes miembros de los equipos de desarrollo distribuidos, para que dicho conocimiento pueda ser utilizado por otros en cualquier momento. 

Siguiendo la l�nea de gesti�n del conocimiento, el objetivo que se pretende alcanzar es el desarrollo de una herramienta que permita registrar, almacenar y visualizar el conocimiento en las fases de an�lisis y dise�o de un producto software. Utilizando el enfoque de \textit{Design Rationale} \cite{libro}, el conocimiento que se desea capturar son las decisiones de dise�o que se toman en esas fases de desarrollo del producto, junto con el razonamiento que se lleva a cabo (planteamiento de preguntas, ideas, opiniones, etc.) para tomar dichas decisiones. \\
\indent De este modo, los diferentes equipos de desarrollo que se encuentran distribuidos pueden continuar su trabajo a partir de las decisiones de dise�o ya tomadas por otros, por lo que es una manera de sincronizar el desarrollo. Inclusive, se pueden plantear problemas o cuestiones, que pueden ser debatidas entre diferentes equipos de desarrollo hasta llegar a un consenso, por lo que tambi�n es una forma de comunicaci�n entre dichos equipos.

Para terminar, cabe se�alar que la herramienta se integrar� en el IDE Eclipse \cite{Eclipse} (en forma de plug-in), para facilitar el registro y visualizaci�n de las decisiones de dise�o tomadas en las etapas de an�lisis y dise�o.