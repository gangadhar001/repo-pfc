\subsection{Quinta iteraci�n} 

En esta �ltima iteraci�n de la fase de construcci�n, se dise�an, implementan y prueban los casos de uso que componen la funcionalidad de \textit{Exportar conocimiento}, mostrados en la Figura \ref{fig:CduExportarConocimiento2Server}, y la funcionalidad de \textit{Gesti�n de idiomas} (ver Figura \ref{fig:CduGestionIdiomas2Server}).

\subsubsection{Funcionalidad de \textit{Exportar conocimiento}} \label{sec:exportarConocimiento}

\paragraph{Diagramas de secuencia}

En la Figura \ref{fig:secuenciaExportarConocimientoCliente} se muestra el diagrama de secuencia para el caso de uso \textit{Adjuntar ficheros} en el cliente, mientras que en la Figura \ref{fig:secuenciaExportarConocimientoServer} se muestra el diagrama de secuencia para el servidor.

%\imagen{Cap5/Cliente2//secuenciaConsultarNotificacionCliente}{1.0}{Diagrama de secuencia - Cliente - Consultar notificaciones ficheros}{fig:secuenciaConsultarNotificacionCliente}

%\imagen{Cap5/Server2//secuenciaConsultarNotificacionServer}{1.0}{Diagrama de secuencia - Servidor - Consultar notificaciones}{fig:secuenciaConsultarNotificacionServer}



\paragraph{Dise�o e implementaci�n}

\subparagraph{Servidor}

En el dise�o e implementaci�n de esta funcionalidad, cabe destacar la utilizaci�n de \textbf{JAXB} (ver secci�n \ref{sec:marco}) para serializar las clases que componen la jerarqu�a de decisiones y toda su informaci�n relacionada (ver Figura \ref{fig:clasesKnowledge}) a un archivo XML. Para ello, se utilizan anotaciones sobre las clases, indicando que esa clase y sus atributos deben convertirse a un nodo XML. Adem�s, es necesario que exista una clase que represente al nodo ra�z del fichero XML y que, por tanto, contenga toda la lista de decisiones. Dicha clase, como ya se ha comentado anteriormente, es la clase llamada \textit{TopicWrapper}.

En el fragmento de c�digo \ref{list:JAXBTopics} se muestran las anotaciones de JAXB realizadas sobre la clase \textit{TopicWrapper}, que engloba a todos los \textit{topics} de un proyecto. En el fragmento de c�digo \ref{list:JAXBKnowledge} se muestran las anotaciones de JAXB para la clase \textit{Knowledge}, para poder serializar sus atributos. Del mismo modo se realizan las anotaciones en el resto de clases que componen el diagrama de clases mostrado en la Figura \ref{fig:clasesKnowledge}.

\texttt{\lstinputlisting[caption=Anotaciones de JAXB sobre la clase \textit{TopicWrapper}, breaklines=true, label=list:JAXBTopics, inputencoding=latin1, style=JavaStyle]{Codigo//JAXBTopics.java}}

\texttt{\lstinputlisting[caption=Anotaciones de JAXB sobre la clase \textit{Knowledge}, breaklines=true, label=list:JAXBKnowledge, inputencoding=latin1, style=JavaStyle]{Codigo//JAXBKnowledge.java}}


%% FALTA CLIENTE


\paragraph{Pruebas}



\subsubsection{Funcionalidad de \textit{Gesti�n de idiomas}}

\paragraph{Diagramas de secuencia}

En la Figura \ref{fig:secuenciaCambiarIdiomaCliente} se muestra el diagrama de secuencia para el caso de uso \textit{Adjuntar ficheros} en el cliente, mientras que en la Figura \ref{fig:secuenciaCambiarIdiomaServer} se muestra el diagrama de secuencia para el servidor.


\imagen{Cap5/Cliente2//secuenciaCambiarIdiomaCliente}{0.6}{Diagrama de secuencia - Cliente - Gesti�n de idiomas}{fig:secuenciaCambiarIdiomaCliente}

\imagen{Cap5/Server2//secuenciaCambiarIdiomaServer}{2.0}{Diagrama de secuencia - Servidor - Gesti�n de idiomas}{fig:secuenciaCambiarIdiomaServer}


\paragraph{Dise�o e implementaci�n}

Ambos subsistemas, tanto el cliente como el servidor, deben soportar multi-idioma, por lo que se ha dise�ado e implementado un gestor de lenguajes. Dicho gestor ser� el encargado de, seg�n el idioma elegido en la aplicaci�n, mostrar la diferente informaci�n de la interfaz gr�fica de usuario en ese idioma. Para ello, se han creado ficheros de propiedades (o \textit{properties)} que contienen las cadenas de texto a internacionalizar, utilizando uno u otro seg�n el idioma escogido. El nombre de dichos ficheros terminan con el c�digo del idioma de cada pa�s, para poder cargar uno u otro seg�n el idioma.

En el fragmento de c�digo \ref{list:internacionalizacion} se muestra c�mo utilizar los archivos \textit{properties} para dar soporte al multi-idioma.

\texttt{\lstinputlisting[caption=Soporte multi-idioma, breaklines=true, label=list:internacionalizacion, inputencoding=latin1, style=JavaStyle]{Codigo//internacionalizacion.java}}

Para terminar, se�alar que los idiomas disponibles para configurar la aplicaci�n se encuentran un fichero XML, para que no sea necesario una conexi�n a base de datos y se pueda modificar el idioma de la aplicaci�n sin tener que acceder previamente a dicha base de datos. Adem�s, se facilita la extensibilidad y adici�n de nuevos idiomas, ya que s�lo abr�a que crear su fichero de propiedades y agregar ese idioma al fichero XML.


\paragraph{Pruebas}
