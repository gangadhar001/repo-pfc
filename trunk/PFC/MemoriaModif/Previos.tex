% Se debe recordar que aunque en el tipo de documento report se puede incluir un abstract (resumen). En la clase book este entorno no est� definido.
%====================================
%====================================
%.... RESUMEN (opcional, 1 p�g.)
\selectlanguage{spanish}%
\begin{abstract}

En las �ltimas d�cadas, el fen�meno de la globalizaci�n est� afectando al modelo de negocio de las empresas, evolucionando hacia un mercado global que busca reducir los costes, aumentar la productividad y la ventaja competitiva. \\
\indent Las empresas dedicadas al desarrollo de software no son ajenas a este fen�meno, y tambi�n se est�n adaptando para desarrollar el software de manera distribuida, en diferentes equipos de desarrollo dispersos por cualquier pa�s del mundo. Es lo que se conoce como \textit{Desarrollo Global de Software (DGS)}. \\
\indent Este paradigma de desarrollo software introduce un gran n�mero de ventajas para las empresas que lo siguen, pero tambi�n introduce una serie de dificultades y desaf�os, asociados a las distancias geogr�ficas, temporales y socio-culturales. Una de las principales dificultades aparece con la Gesti�n del Conocimiento y de Decisiones, ya que se produce informaci�n desde muchas y diversas fuentes, lo que dificulta su gesti�n, almacenamiento y reutilizaci�n. \\
\indent Para paliar algunos de estos desaf�os, se propone desarrollar un sistema que permita dar soporte a la gesti�n de decisiones en proyectos software en el contexto de desarrollo global. Por tanto, dicho sistema debe permitir la creaci�n, almacenamiento, recuperaci�n y transmisi�n de decisiones abordadas en un proyecto software, realizado de manera deslocalizada. Adem�s, debe permitir gestionar
tambi�n los proyectos software y reutilizar las decisiones tomadas en proyectos previos en nuevos proyectos con caracter�sticas similares. 

\end{abstract}

%====================================
%====================================
%.... ABSTRACT (opcional, 1 p�g.)
% Aqu� se muestra un ejemplo completo en el que se a�ade una versi�n en ingl�s del resumen.
\selectlanguage{english}% Selecci�n de idioma ingl�s.
\begin{abstract}
From recent decades, the phenomenon of globalization is affecting the business model of companies, evolving into a global market that seeks to reduce costs, increase productivity and competitive advantage. \\
\indent The companies engaged in software development are no strangers to this phenomenon, and also being adapted to develop the software in a distributed way at different development teams scattered around the world. This is known as \textit{Global Software Development (GSD)}. \\
\indent This software development paradigm introduces a number of advantages for companies that follow it, but also introduces a number of difficulties and challenges associated with geographical, temporal and socio-cultural distances. One of the major difficulty appears in the Knowledge and Decisions Management, because it produces information from many different sources, which makes its management, storage and reuse very complicated. \\
\indent In order to mitigate some of these challenges, it intends to develop a system to support the decisions management made in software projects, in the context of global development. Therefore, the system should enable the creation, storage, retrieval and transmission of decisions tackled in a software project, carried out in a delocalized way. In addition, the system should allow also manage software projects and permit reuse the decisions taken in previous projects into new projects with similar characteristics.
\end{abstract}

\selectlanguage{spanish}% Para el resto del documento del idioma es espa�ol.
%====================================
%====================================
%.... AGRADECIMIENTOS (opcional, 1 p�g. o m�s, recomendable hacerloe en una s�lo)
% Los agradecimientos son m�s extensos que la dedicatoria como se muestra en el ejemplo.
\chapter*{Agradecimientos} % Opci�n con * para que no aparezca en TOC
Escribir agredecimientos