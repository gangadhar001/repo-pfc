\chapter{Manual de usuario}
\label{ch:manualServer}

\textit{DPMTool - Server} es el sistema servidor que debe encontrarse en ejecuci�n para poder ejecutar la pr�ctica totalidad de las funcionalidades del sistema  \textit{DMPTool - Client}. 

Dicho servidor debe contar con un servidor de base de datos MySQL activo, bien en la misma m�quina o en otra. En el ap�ndice \ref{ch:manualEntorno} se detallan los requisitos del sistema y se explica como instalar y configurar el entorno. 

\section{Ventana principal}

Al ejecutar el fichero JAR correspondiente al servidor, obtenemos una ventana como la que se muestra en la figura \ref{fig:server-ventanaprincipal}. Si la describimos de arriba abajo se aprecian los siguientes elementos:
\begin{itemize}
	\item Una barra de men�s (\textit{Archivo}, \textit{Opciones} y \textit{Ayuda}).
	\item Una toolbar con botones para poner el servidor a la escucha, pararlo o cerrarlo.
	\item Un �rea de texto donde se va mostrando el log de las acciones que realizan los clientes sobre el servidor.
	\item Una barra de estado donde se muestra informaci�n relativa a las comunicaciones: clientes conectados, datos de conexi�n del servidor, datos de conexi�n del sistema gestor de bases de datos, y datos de conexi�n del servidor de respaldo.
\end{itemize}

%\imagen{Apendices//server-ventanaprincipal}{0.70}{Ventana principal del servidor}{fig:server-ventanaprincipal}

\section{Configurar el servidor}

Al seleccionar el men� \textit{Opciones} -\textgreater \textit{Configurar} aparecer� una ventana para configurar el servidor, es decir, los datos de conexi�n del SGBD y el puerto en que el servidor deber� ponerse a la escucha. Para confirmar los cambios, haga clic en \textit{Aceptar} (ver Figura \ref{fig:server-configurar}).

%\imagen{Apendices//server-configurar}{0.70}{Ventana de configuraci�n servidor front-end}{fig:server-configurar}

\section{Cambiar el idioma del servidor}

Al seleccionar el men� \textit{Opciones} -\textgreater \textit{Cambiar idioma} aparecer� una ventana para poder cambiar el idioma de la interfaz y de los mensajes de log mostrados (ver Figura \ref{fig:server-idioma}). El cambio de idioma requerir� el reinicio de la aplicaci�n.

%\imagen{Apendices//server-configurar}{0.70}{Ventana de configuraci�n servidor front-end}{fig:server-configurar}

\section{Activar el servidor}

Para poner el servidor a la escucha, se debe hacer clic en el bot�n \textit{Conectar} (o en su hom�logo de la toolbar). Cuando alg�n cliente se conecte aparecer� reflejado en la barra de estado. Si hay clientes conectados y desea desconectar o cerrar el servidor, se le pedir� confirmaci�n para forzar la desconexi�n de los clientes activos.


\section{Manual de usuario de GCADClient}
\label{ch:manualClient}


