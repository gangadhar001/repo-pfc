\section{Introducci�n} \label{introduccion}

La manera de desarrollar software est� evolucionando en los �ltimos a�os, tendiendo a un desarrollo distribuido del mismo. �sto se debe, en cierta medida, a la globalizaci�n del mercado, de tal modo que se ha cambiado el concepto de \textit{mercado nacional} por el de \textit{mercado global} \cite{herbsleb2002global}. De este modo, el desarrollo del software es llevado a cabo por diferentes empresas, que se encuentran distribuidas en diferentes ciudades, pa�ses o continentes. Es lo que se conoce como Desarrollo Global de Software (DGS).

Una de las principales ventajas del DGS es que se puede aumentar la productividad a la hora de desarrollar el software, ya que la jornada laboral puede ser extendida debido a las diferencias horarias existentes en los diferentes pa�ses donde se distribuyen las empresas de desarrollo. \\
\indent Para aprovechar al m�ximo esta ventaja, es imprescindible una correcta Gesti�n de Conocimiento aplicada al DGS, que permita registrar, almacenar y organizar todo el conocimiento (ideas, decisiones, experiencias, etc.) generado por los diferentes equipos de desarrollo distribuidos, para que dicho conocimiento pueda ser utilizado por otros en cualquier momento y lugar. 

Por otro lado, la deslocalizaci�n de las empresas es uno de los principales inconvenientes del DGS, pues surgen problemas de comunicaci�n y coordinaci�n, debido a las diferencias culturales que pueden provocar malentendidos a la hora de utilizar o generar nuevo conocimiento en el proceso de desarrollo del software. 

Centr�ndose en la Gesti�n de Conocimiento, y para paliar el problema anterior, se propone el desarrollo de una herramienta visual para dar soporte a la Gesti�n de Conocimiento en la fase de an�lisis y dise�o de un producto software, aplic�ndose al Desarrollo Global de Software. \\
\indent De este modo, gracias al uso de esta herramienta, se puede recoger y y visualizar el conocimiento (en este caso, las decisiones tomadas en la fase de an�lisis y dise�o) de una manera visual e intuitiva, evitando los problemas de comunicaci�n y malentendidos debidos a las diferencias culturales de los diferentes pa�ses. 