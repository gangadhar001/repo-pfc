\section{Introducci�n} \label{introduccion}

La manera de desarrollar software est� evolucionando en los �ltimos a�os, tendiendo a un desarrollo distribuido del mismo. �sto se debe, en cierta medida, a la globalizaci�n del mercado, de tal modo que se ha cambiado el concepto de \textit{mercado nacional} por el de \textit{mercado global} \cite{herbsleb2002global} \cite{damian2006guest}. De este modo, el desarrollo del software es llevado a cabo por diferentes empresas, que se encuentran distribuidas en diferentes continentes. Es lo que se conoce como Desarrollo Global de Software (DGS).

Una de las principales ventajas del DGS es que se puede aumentar la productividad a la hora de desarrollar el software, ya que la jornada laboral puede ser extendida debido a las diferencias horarias existentes en los diferentes pa�ses donde se distribuyen las empresas de desarrollo. Otras de las ventajas del DGS es que se pueden abaratar costes en mano de obra de diferentes piases y se puede mejorar la presencia en el mercado internacional, aumentando la competitividad. 

Para aprovechar al m�ximo las ventajas que proporciona el DGS, es imprescindible una correcta Gesti�n de Conocimiento, que permita registrar, almacenar y organizar todo el conocimiento (ideas, decisiones, experiencias, artefactos, etc.) generado por los diferentes equipos de desarrollo distribuidos, para que dicho conocimiento pueda ser utilizado por otros en cualquier momento y lugar \cite{herbsleb2002global}. 

Por otro lado, la deslocalizaci�n de las empresas es uno de los principales inconvenientes del DGS, pues surgen problemas de comunicaci�n y coordinaci�n \cite{herbsleb2002global} \cite{ebert2002surviving} \cite{mohagheghi2004global}, que derivan en problemas a la hora de plantear decisiones y alternativas, llegar a acuerdos, etc.
Adem�s, al estar los equipos de desarrollo distribuidos en diferentes pa�ses, aparecen tambi�n problemas asociados a malentendidos que pueden provocarse por las diferencias culturales, ya que los empleados de las empresas de cada pa�s tendr�n su propia forma de trabajar, el idioma es diferente, etc. \\
\indent Todo ello provoca que sea m�s f�cil cometer errores a la hora de generar nuevo conocimiento asociado al desarrollo de un proyecto, as� como se dificulta tambi�n la utilizaci�n de dicho conocimiento. 

Para intentan paliar los problemas comentados anteriormente, se propone el desarrollo de una herramienta visual para dar soporte a la gesti�n de conocimiento en la fase de an�lisis y dise�o de un producto software, aplic�ndose al desarrollo global de software. \\
\indent De este modo, gracias al uso de esta herramienta, se puede recoger y y visualizar el conocimiento (en este caso, las decisiones tomadas en la fase de an�lisis y dise�o) de una manera visual e intuitiva, evitando los problemas de comunicaci�n y malentendidos debidos a las diferencias culturales entre los diferentes pa�ses. 