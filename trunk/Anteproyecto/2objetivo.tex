\section{Objetivos y fundamentos te�ricos} \label{objetivos}


\indent En esta secci�n se presentar� el objetivo que se pretende alcanzar con el desarrollo del Proyecto Fin de Carrera (en adelante PFC), as� como los fundamentos te�ricos necesario para alcanzar dicho objetivo. 

\subsection{Objetivos del proyecto} \label{objetivoPFC}

\indent El objetivo principal del PFC consistir� en el dise�o y construcci�n de una herramienta visual que permita dar soporte a la gesti�n de decisiones en Desarrollo Global de Software. Concretamente, la herramienta debe dar soporte a las decisiones planteadas en las fases de an�lisis y dise�o de un producto software.

Dicha herramienta se desarrollar� en forma de plug-in para el IDE Eclipse \cite{eclipse}, uno de los m�s utilizados por las empresas en el �mbito de desarrollo de software. 

Como se ha comentado en la secci�n \ref{introduccion}, con esta herramienta visual se intenta reducir o eliminar una de las problem�ticas que aparece en DGS, como es la falta de comunicaci�n y control, as� como los malentendidos que pueden surgir a la hora de tomar decisiones en el desarrollo de un producto software. \\
\indent Por esta raz�n, la herramienta debe cumplir con los siguientes objetivos:

\begin{milista}
	\item Almacenar conocimiento en forma de alternativas y decisiones planteadas en las fases de an�lisis y dise�o de un producto software. 
	\item Mostrar de una forma visual las diferentes alternativas y decisiones que han sido tomadas en las fases de an�lisis y dise�o de un producto software. 
	\item Notificar nuevas contribuciones y conocimiento disponible. 
	\item Proporcionar informaci�n acerca de qui�n aporta una alternativa o decisi�n (rol de esa persona, pa�s donde se localiza, experiencia acumulada en desarrollo de software, antig�edad en la empresa, etc.).
	\item Aconsejar decisiones de an�lisis y dise�o que hayan sido exitosas en otros proyectos con caracter�sticas similares. 
	\item Informar sobre la contribuci�n de cada uno de los miembros del proyecto, ofreciendo estad�sticas. 
\end{milista}


\subsection{Fundamentos te�ricos}

Los conceptos te�ricos que intervienen en este PFC son el Desarrollo Global de Software y la Gesti�n de Conocimiento. El primero de ellos ya ha sido comentado en la secci�n \ref{introduccion}, por lo que se presenta a continuaci�n el concepto de Gesti�n de Conocimiento.

\paragraph{Gesti�n del Conocimiento}

