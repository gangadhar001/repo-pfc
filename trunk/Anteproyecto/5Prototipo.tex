\section{Prototipo}

En esta secci�n se presenta unas im�genes de algunas funcionalidades de la herramienta que se est� desarrollando.

Para comenzar a utilizar esta herramienta, la primera acci�n que hay que realizar es identificarse en el sistema, conectando a una base de datos. La Figura \ref{fig:login} muestra el asistente (\textit{wizard}) utilizado para ello. 

\imagen{loginWizard}{0.75}{\textit{Wizard} para identificarse en el sistema}{fig:login}

Una vez identificado en el sistema, se pueden utilizar los men�s de la herramienta (ver Figura \ref{fig:tool}) para introducir nuevo conocimiento (decisiones de dise�o). Seg�n el rol del usuario que se ha identificado, estar� permitido unas acciones u otras. Adem�s de utilizar el men�, se pueden utilizar las acciones en la barra de herramientas de las vistas correspondientes o haciendo doble clic en un nodo de la vista de grafo o de la vista de �rbol. En la Figura \ref{fig:proposal} se muestra un ejemplo del asistente utilizado para insertar una nueva propuesta a un tema (\textit{"`topic"'}) existente.

\imagen{perspective}{0.35}{Diferentes vistas y men�s de la herramienta}{fig:tool}

\imagen{newProposal}{0.75}{\textit{Wizard} para introducir una nueva propuesta}{fig:proposal}

Como se puede observar en la Figura \ref{fig:tool}, esta herramienta constar� de varias vistas, agrupadas bajo una perspectiva. Las vistas que aparecen en dicha figura muestran de manera visual el contenido de la base de datos, as� como gr�ficos que informan sobre la contribuci�n de cada persona en el proyecto.

Para terminar, cabe destacar que la aplicaci�n se encuentra actualmente en construcci�n, por lo que a�n se tienen que a�adir nuevas funcionalidades y detalles. A continuaci�n se listan algunas de ellas:

\begin{milista}
	\item Notificaciones para informar al usuario que inicia sesi�n de que existe nuevo conocimiento disponible en el proyecto.
	\item Vista para mostrar informaci�n acerca de qu� usuario ha propuesto conocimiento.
	\item Vista de estad�sticas, informando de las contribuciones de los diferentes empleados de cada proyecto.
	\item Mejora de la usabilidad de la herramienta, utilizando, por ejemplo, iconos y met�foras gr�ficas.
\end{milista}

