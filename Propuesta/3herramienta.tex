\section{Herramientas existentes} \label{existentes}

Una vez planteado el tipo de herramienta que se desea desarrollar y el objetivo que se pretende alcanzar (ver secci�n \ref{objetivo}), se ha investigado sobre otras herramientas existentes en el �mbito de la gesti�n de conocimiento en ingenier�a de software, destacando las que se mencionan en los siguientes apartados. 

\paragraph{Stylebase} 

Stylebase \cite{stylebase} es un plug-in de Eclipse que mantiene una base de conocimiento arquitect�nico, almacenando diferentes patrones de dise�o, arquitect�nicos, etc. El objetivo de la herramienta es aumentar la informaci�n que se comparte y se reutiliza en equipos de desarrollo distribuidos.

\paragraph{SADL-IDE} 

\textit{Semantic Application Design Language} (SADL) \cite{sadl} es un lenguaje que permite crear reglas y modelos sem�nticos para capturar conocimiento del dominio de un sistema software. La herramienta ''SADL-IDE'' es un plug-in de Eclipse que permite editar y testear modelos creados utilizando el lenguaje SADL. 

\paragraph{Work item tracking} 

Esta herramienta, que puede integrarse en IDEs como Eclipse o Visual Studio, o utilizarse a trav�s de la Web, permite crear y editar elementos de trabajo (\textit{work items}) \cite{work}. A trav�s de estos elementos de trabajo se pueden compartir im�genes, plantear cuestiones, asociar problemas a diferentes categor�as, enviar mensajes directos a un usuario (utilizando el s�mbolo arroba ''\@''), etc.

\paragraph{SEURAT} SEURAT (\textit{Software Engineering Using RATionale}) \cite{seurat}, \cite{libroSeurat} es un plug-in de Eclipse que da soporte al uso de \textit{Rationale} en la fase de mantenimiento del software. De este modo, recoge y visualiza problemas e inconsistencias del software, asoci�ndolas al c�digo fuente para facilitar su mantenimiento.

\paragraph{Archium} Archium \cite{archium} es un prototipo que permite capturar decisiones arquitect�nicas que se toman en el desarrollo de un sistema software. Utiliza un lenguaje llamado ADL (\textit{Architectural Description Language}) que se integra con Java y permite describir las decisiones arquitect�nicas.



